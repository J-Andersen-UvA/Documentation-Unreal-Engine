%-------------------------------------------------------------------------------
% LATEX TEMPLATE ARTIKEL
%-------------------------------------------------------------------------------
% Dit template is voor gebruik door studenten van de de bacheloropleiding 
% Informatica van de Universiteit van Amsterdam.
% Voor informatie over schrijfvaardigheden, zie 
%                               https://practicumav.nl/schrijven/index.html
%
%-------------------------------------------------------------------------------
%	PACKAGES EN DOCUMENT CONFIGURATIE
%-------------------------------------------------------------------------------

\documentclass{uva-inf-article}
\usepackage[english]{babel}
\usepackage{tikz}
\usepackage{pdflscape}
\usepackage{todonotes}
\usepackage{listings}
\lstset{
basicstyle=\small\ttfamily,
columns=flexible,
breaklines=true
}


\usepackage[style=authoryear-comp]{biblatex}
\addbibresource{references.bib}

%-------------------------------------------------------------------------------
%	GEGEVENS VOOR IN DE TITEL, HEADER EN FOOTER
%-------------------------------------------------------------------------------

% Geef je artikel een logische titel die de inhoud dekt.
\title{Unreal Engine Documentation}

% Vul de naam van de opdracht in zoals gegeven door de docent en het type 
% opdracht, bijvoorbeeld 'technisch rapport' of 'essay'.
% \assignment{}
% \assignmenttype{}

% Vul de volledige namen van alle auteurs in en de corresponderende UvAnetID's.
\authors{Jari Andersen}
% \uvanetids{}

% Vul de naam van je tutor, begeleider (mentor), of docent / vakcoördinator in.
% Vermeld in ieder geval de naam van diegene die het artikel nakijkt!
% \tutor{}
% \mentor{}
% \docent{}

% Vul hier de naam van je tutorgroep, werkgroep, of practicumgroep in.
% \group{SignLab, VisualisationLab}

% Vul de naam van de cursus in en de cursuscode, te vinden op o.a. DataNose.
% \course{}
% \courseid{}

% Dit is de datum die op het document komt te staan. Standaard is dat vandaag.
\date{\today}

%-------------------------------------------------------------------------------
%	VOORPAGINA 
%-------------------------------------------------------------------------------

\begin{document}
\maketitle

%-------------------------------------------------------------------------------
%	INHOUDSOPGAVE EN ABSTRACT
%-------------------------------------------------------------------------------
% Niet toevoegen bij een kort artikel, zeg minder dan 10 pagina's!

%TC:ignore
\tableofcontents
%\begin{abstract}
%\end{abstract}
%TC:endignore
\newpage
%-------------------------------------------------------------------------------
%	INHOUD
%-------------------------------------------------------------------------------
% Hanteer bij benadering IMRAD: Introduction, Method, Results, Discussion.
\section{Introduction}
Welcome to this comprehensive documentation compiled during the development journey in Unreal Engine. Throughout this document, you'll find information encompassing various aspects of development, ranging from the utilization of C++, Blueprints, and Python to defining structs for data tables. We'll also explore best practices and strategies for implementing Git, ensuring efficient collaboration and version control. Mostly, we'll touch upon bugs that we've encountered and hopefully fixed. This documentation aims to provide valuable insights and practical guidance for Unreal Engine projects.

\subsection{Live Link}
For the Live Link Pipeline documentation, go to the Unreal Live Link Pipeline documentation repository (\url{https://github.com/J-Andersen-UvA/LiveLinkPipeline-UnrealEngine}).

% \begin{landscape}
% \vspace*{\fill}
% \begin{figure}[hbt!]
%     \centering
%     \includegraphics[width=1.5\textheight]{imgs/pipeline16-2-24.png}
%     \caption{Current Live Link Pipeline. Three streams into Unreal Engine}
%     \label{fig:pipeline}
% \end{figure}
% \vspace*{\fill}
% \end{landscape}

\section{Compiling and building C++ projects}\label{CompilingAndBuilding}
When it comes to building the C++ files in Unreal we either use the IDE or Unreal Engine. Unreal Engine's compilation system comes with a functionality called Live Coding. Live coding with "hot reloading" allows developers to make changes to their C++ code while the game is running and see the changes reflected in real-time without the need to restart the game or editor. For our purposes (and because it causes many bugs), we disable the Live Coding option (see Figure \ref{fig:liveCoding} on where to disable this, and Section \ref{datatableBug} on an example bug caused by Live Coding).
\begin{figure}[hbt!]
    \centering
    \includegraphics{imgs/livecoding.png}
    \caption{Where to disable Live Coding}
    \label{fig:liveCoding}
\end{figure}

Now we can compile in Unreal Engine or in an IDE like Visual Studio. Figure \ref{fig:vstudio} shows where to build using Visual Studio. Do make sure to follow the instructions in the following link to setup the Build tool first: \url{https://docs.unrealengine.com/4.27/en-US/ProductionPipelines/DevelopmentSetup/BuildingUnrealEngine/}. As for Unreal Engine, use the button showed in Figure \ref{fig:liveCoding}.

\begin{figure}[hbt!]
    \centering
    \includegraphics{imgs/buildingInVStudio.png}
    \caption{Where to build using Visual Studio}
    \label{fig:vstudio}
\end{figure}
\section{Git Version control}
We use Git for Unreal Engine to efficiently manage project versions, and to enable collaboration among team members. Git allows us to track changes to code, blueprints, and assets. By utilizing Git, Unreal Engine developers can maintain organized workflows, iterate rapidly, and coordinate effectively on complex projects.

\subsection{How to install and use Git for Unreal Engine}
Adding project to a blank repository Git:
\begin{enumerate}
    \item In the bottom right of a level screen in Unreal Engine, click revision control.
    \item Select connect to revision control.
    \item After selecting Git, add the url of your repository and initialize.
\end{enumerate}

Now you can use the same revision control button to add changes, write commits and view changes. You do however, need to push the commits by hand.
Note that sometimes some files aren't added to the Git, you'll have to fix that by adding them by hand.

\subsection{Local Plugins}
In Unreal Engine, third-party plugins play a significant role in extending functionality and enhancing project capabilities. However, unlike native plugins bundled within the Engine plugins folder, third-party plugins must be added to the project's root plugins folder due to the engine's architecture.

When we add plugins locally however, the plugins will be pushed with the rest of the project. This can cause bugs later on (see Section \ref{cppThirdPartyPluginsBug}). Especially for cloning the repository on a new machine this causes issues (follow Section \ref{cloning} for best practices on cloning a repository).

\subsection{Cloning}\label{cloning}
When cloning a repository everything usually works well. There are cases however, when cloning a repository breaks the build (for example \ref{cppThirdPartyPluginsBug}).
If we open a new project that is broken, Unreal Engine will prompt us if he can rebuild it. This works sometimes, but in cases where a local plugin prevents us to build we can do a simple trick:
\begin{itemize}
    \item Move the local plugin out of the project.
    \item Rebuild in your IDE.
    \item Open the project.
    \item Put the plugin back and rebuild.
\end{itemize}

\section{Using C++, Python, and Blueprints}
After fighting a bit in Unreal with Python, C++, and Blueprints, we have formed thoughts on how to use and when to apply these tools. For example, C++ has nice integration with Blueprints, and it seems to be the normal way of coding in Unreal Engine. We have seen other code bases that integrate both. Therefore, we will not migrate everything to C++, but we will integrate C++ into our workflow for Unreal Engine. We can say similar things about Python. It has its own strengths, especially when coupling it with our other code that already use sockets to communicate.

\subsection{Python executing editor functionalities in Unreal Engine}\label{PyUE}
In Unreal Engine, many API calls, especially those related to the game world, UI updates, and editor functions, must be called from the main thread. This makes it difficult to use a waiting/listening Python program. If we wait on the main thread, the entire engine stalls making it impossible to wait for things to execute in the engine and wait for it. There is a workaround however. In Unreal Engine Python scripting, especially in editor contexts, hooking into the tick method (such as using register\_slate\_post\_tick\_callback) is a common approach for executing continuous or periodic tasks. This approach is necessary due to the nature of Unreal Engine's architecture and threading model.
Creating a new Python thread to handle functionality might seem like a simpler solution, but it poses several issues in the Unreal Engine environment:
\begin{itemize}
    \item Thread safety: Unreal Engine is not thread-safe by default. Many of its APIs cannot be called safely from a separate thread. This includes accessing game objects, modifying the world, and interacting with the editor UI.
    \item Blocking operations: If you use blocking calls (e.g., waiting for a network response) on the main thread, it would freeze the UI and the entire editor/game. Conversely, if you use a new thread for non-blocking operations, you still need a way to synchronize the results back to the main thread.
\end{itemize}

\textbf{Note:} using the tick function comes with its own difficulties. For example, a file saved on a tick is not always accessible by Unreal Engine's API in the same tick. It seems that the file structure is only updated after a tick. Please keep un-intuitive issues like this in mind when coding with Python in Unreal Engine.

\subsection{Combining Python and Blueprints}
Not all functionalities are easily accessible through Python, and sometimes you don't want to bother with setting up an entire tick hook either. Some implementations of Unreal Engine functionalities are easier through Blueprints. What if we want the advantages of Blueprints and Python? We have two options:
\begin{itemize}
    \item Python nodes within Blueprints.\\
    If you have the Python Editor Script plugin enabled, you should be able to see Python nodes exclusively within Utility blueprints only - not regular blueprints. Editor utility blueprints, widgets, or even action asset utilities support the Python nodes. Python is unavailable in regular blueprints unfortunately. 
    \item Event calls to Blueprint events within Python.\\
    If you have an actor added to the scene, and you have loaded the actor onto a variable in Python using Unreal Python functions such as \textit{unreal.EditorLevelLibrary.get\_all\_level\_actors()} you can access any method/custom-event in its blueprint using the following code: \textit{replay\_actor.call\_method("methodName", args=("argument1",))}
\end{itemize}

\subsection{Installing packages with pip and UE}
To install Python packages in Unreal Engine, use the following command in the terminal, replacing $<$package$>$ with the desired package name:
"C:\textbackslash Program Files\textbackslash Epic Games\textbackslash UE\_5.5\textbackslash Engine \textbackslash Binaries\textbackslash ThirdParty\textbackslash Python3\textbackslash Win64\textbackslash python.exe" -m pip install $<$package$>$
For our own Python code, you need to install the following packages:
\begin{itemize}
    \item requests
    \item httpx
    \item pyyaml
    \item websocket
    \item websockets
    \item websocket-client
    \item flask
    \item flask\_cors
\end{itemize}
Simply run the command for each package to ensure proper functionality in your UE project.


\section{Animation editing}
Sometimes, motioncaptured animations need to be altered by hand. This section goes over how we alter animations within Unreal Engine. After importing the animations into Unreal Engine, we will the fix the animation within Unreal Engine's ``animation sequencer" (figure \ref{fig:animationsequencer}). Since the animations are baked onto the skeleton, the process of fixing the animation requires us to alter the orientation and position of the individual bones themselves. This method offers a more straightforward means of intervention compared to modifying marker data. When editing marker data, alterations become evident only after the animation is re-solved. In contrast, when addressing problematic frames and bones in this context, we can directly modify the animation. Because the alterations will be additive onto the rig, we are able to add new movements onto a bone without having to redo the entire animation. Note however, that not every animation issue can be solved with the animation sequencer. For example, extreme jitters would take too long to fix by hand.

Following is a step-by-step guide on altering animations through Unreal Engine's animations sequencer:
\begin{enumerate}
    \item Either open the animation by selecting it in the \say{level sequencer}, or open the animation and in the top \say{edit in sequencer} drop-down menu select \say{Edit With FK Control Rig}. Both sequencers are the same, but the latter doesn't require you to add the avatar to the scene by hand. For the first option make sure to bake the animation onto a FK Control Rig by right clicking the skeleton in the sequencer and selecting the \say{Edit With FK Control Rig} option.
    \item In the sequencer, add an additive section to the \say{FKControlRig} in the sequencer.
    \item Beginning with the first frame displaying odd behaviour, select the bone closest to the root displaying the abnormality or which you want to alter and create a new key-frame on the additive section in the control rig. Reposition and rotate the bone to align it with the original actor's pose. Repeat this process for all child bones. Following this, inspect the final frame of the anomalous movement. The modifications made to the initial abnormal frame will propagate throughout the animation. To restrict these alterations solely to the relevant portion, insert an additional key-frame at this specific frame and reset the additive movements. Subsequently, by adjusting the curve between these two key-frames, we can make the alterations look more natural. If the result does not appear sufficiently natural, further key-frames may be required either within or outside the altered segment to achieve a smoother transition of the adjustments. Similar to data cleaning with Shogun Post, this task is typically performed by experienced professionals, and it may require some level of practice and proficiency.
    \item Finally, when the results are adequate, right click on the skeleton in the level sequencer and select ``Bake Animation Sequence".
\end{enumerate}

\begin{figure}[hbt!]
    \centering
    \includegraphics[width=.8\textwidth]{imgs/CleaningUpAnAnimation.png}
    \caption{The animation sequencer with an example frame being altered by using bone rotations and key-frames.}
    \label{fig:animationsequencer}
\end{figure}

\subsection{Animation bone group Blending using the sequencer}\label{sec:bonegrouping}
With multiple animations we have the ability to blend between them and over each other, but what if we want to do that only for a specific bone group? Using ``Bone Grouping" or ``Animation Layering" we can force animations to play only on their relative group. Documentation wise, there is not a whole lot to go of of in Unreal Engine. Many people have made posts and videos about this, but after searching for a long while I was only able to find a single post originating from Unreal Engine 4 that goes over this topic in conjunction with the Level Sequencer. Which is exactly what we want if we want to alter animations by hand. The following link goes over this, so please have a read: \url{https://forums.unrealengine.com/t/is-it-possible-to-use-multiple-animation-slots-in-sequencer-to-layer-animations/427244/2}. In order to showcase how to do this as simple as possible and to add more difficult blendings afterwards, I have created the following step by step guide:
\begin{enumerate}
    \item Create an animation blueprint on the skeletalMesh of the avatar where you want to apply the blend.
    \item In the Anim Slot Manager tab (can be found either in the anim graph using the window options, or in the animation itself) add an Anim Slot with the name of the parent bone which hierarchy will be grouped. In the current example, ``RightHand" is used.
    \item In the anim graph add the following:
    \begin{itemize}
        \item 2 slot ``DefaultSlot" nodes.\\Change one name to the corresponding Anim Slot to group (for example ``RightHand").
        \item A layered blend per bone node.\\In the Details tab, add a Layer with the same name as the Anim Slot (for example ``RightHand").
    \end{itemize}
    If you have applied these steps, the resulting graph will look like the one in figure \ref{fig:boneGrouping}.
    \begin{figure}[hbt!]
        \centering
        \includegraphics[width=.9\textwidth]{imgs/animgraphsimple.png}
        \caption{The simplest animation graph that applies bone grouping.}
        \label{fig:boneGrouping}
    \end{figure}
    \item Compile the blueprint.
    \item Now in the normal view (world/level view) in Unreal Engine, drag the animation blueprint into the level.
    \item Open the Sequencer.
    \item Add the dragged animation blueprint to the sequencer.
    \item Add the 2 animations to be grouped (see figure \ref{fig:animSeq}).
    \begin{figure}[hbt!]
        \centering
        \includegraphics[width=.9\textwidth]{imgs/animationSequencer.png}
        \caption{Adding an animation to the sequencer.}
        \label{fig:animSeq}
    \end{figure}
    \item Right click the animation that we want to group, go to properties, then type in the name of the group it should exist on (see figure \ref{fig:boneGroupPreference}).
    \begin{figure}[hbt!]
        \centering
        \includegraphics[width=.9\textwidth]{imgs/bonegroup.png}
        \caption{The location of the bonegroup option in the sequencer.}
        \label{fig:boneGroupPreference}
    \end{figure}
\end{enumerate}
\textit{Hint, we can define as many animation slots/bone groups as we want by adding layers to the layered blend per bone node.}

Congratulations, this is how we layer blends per bone in the level sequencer.
Now, if we want to take this further and for example blend the animation on the RightHand using 2 animations we can do the following:
\begin{enumerate}
    \item Add another animation to the sequencer.
    \item Define the animation slot in the properties menu (see figure \ref{fig:boneGroupPreference}).
    \item Tweak the weights between the 2 animations playing on the animation slot using the ``Weight" option in the level sequencer (see figure \ref{fig:weights}).
    \begin{figure}[hbt!]
        \centering
        \includegraphics[width=.9\textwidth]{imgs/weights.png}
        \caption{The location of the weights option in the sequencer.}
        \label{fig:weights}
    \end{figure}
\end{enumerate}

\section{Driving blendshapes through the level sequence}
This section is also discussed in the UnrealEngineLiveLinkPipeline documentation.

Using the recordings for Live Link Face from the IPhone itself (.csv), we can reenact a Live Link Face stream by applying the ``LiveLinkFaceImporter" plugin. For a video explanation follow this link: \url{https://www.youtube.com/watch?v=TKd4iSEbWNI}.
When scrubbing through the imported level sequence we can observe in the Live Link tab that a new stream is created. We can catch this stream like any other live link stream and ennact upon it (see figure \ref{fig:llfimporter}).
\begin{figure}[hbt!]
    \centering
    \includegraphics[width=.9\textwidth]{imgs/livelinkfaceimporter.png}
    \caption{Scrubbing through the imported .csv level sequence from Live Link Face.}
    \label{fig:llfimporter}
\end{figure}

\subsection{Rendering Live Link sourced (blendshape) animations}
When rendering we will not have the ability to simulate a Live Link Source. Therefore, any animation you see through the sequencer using this source will not be outputted on the avatar. To circumvent this we can bake the animations before rendering and using that animation to render.
\begin{enumerate}
    \item Make a sequence with the Live Link CSV sequence you want and the avatar with the animation blueprint that reads the Live Link Source.
    \item Confirm that the CSV animations are working by scrubbing the sequence.
    \item Right click in the level sequence on the correct skeletal mesh that we need to extract the animations from.
    \item Click ``Bake Animation Sequence" and save the new animation.
    \item Make a new sequence or alter the current sequence using this animation and you should be able to render as usual.
\end{enumerate}

\section{Rendering level sequences}
When we have created a level sequence that we are happy with, we have the ability to render the sequence as a video or frame sequence. To this end we use the ``Movie Render Queue" plugin, so make sure that is enabled. Then add a camera to the sequencer using the ``Create camera" option in the sequencer (see figure \ref{fig:createCam}). In the details panel you can alter how the camera is configured. Next, render the video by selecting the ``Render Movie" option next to the Create camera option. The Movie Render Queue plugin should open, and here you can alter how the movie should be rendered. This is personal preference, but I took the settings from the following video as a baseline \url{https://www.youtube.com/watch?v=JchOd6vGAy0}. Some formats do not immediately render to a viewable file. For example, .exr will result in a whole sequence of files. We can take these type of formats into video editing software such as Davinci resolve to further alter the appearance of the video before rendering the final video there.

\begin{figure}[hbt!]
    \centering
    \includegraphics[width=.9\textwidth]{imgs/addcamera.png}
    \caption{The location of the create camera option in the sequencer.}
    \label{fig:createCam}
\end{figure}

\subsection{Blurry videos}
Make sure the camera is not set to tracking an individual avatar for the ``focus setting". Because our actors are usually stationary and the camera is very close, this will result in very blurry videos. Just set the camera to have no focus method.

\subsection{Motion Blur}
A weird bug is that motion blur is on by default even though the camera settings tell you that it is not. To fix this, turn on motion blur and set the amount to 0.

\subsection{Dark videos}
Looking through the camera view we are deceived by how the result will turn out. Often the resulting videos are much darker than what we view through the camera. To fix this, either use post-processing, brighten up the scene, or in Davinci resolve up the gamma. It is also often the case that some formats are handled differently in different software. EXR for example, is much darker in DaVinci resolve (\url{https://creativecow.net/forums/thread/openexr-way-too-dark-in-davinci/}).


\section{Bugs and fixes}
In the "Bugs and Fixes" section, we address challenges we encountered in Unreal Engine development hopefully along with their solutions.

\subsection{C++ projects and third party plugins / Migrating projects / Cannot build project}\label{cppThirdPartyPluginsBug}
\subsubsection{Cannot build step 1} \url{https://www.youtube.com/watch?v=6mFhtvdYQWI}
\subsubsection{Third party plugins}
Using third party plugins enables us to focus on our own development. It does come with its own challenges, however. One of which we encountered when migrating to C++ based projects. When trying to compile a project that uses third party plugins from the Engine folder that are not in the marketplace we get reference errors, example:
\begin{lstlisting}
Expecting to find a type to be declared in a module rules named 'MoCapProLiveLink' in UE5Rules, Version=0.0.0.0, Culture=neutral, PublicKeyToken=null.  This type must derive from the 'ModuleRules' type defined by Unreal Build Tool.
\end{lstlisting}
Most tips on the internet tell us to move the plugins to a project root folder called Plugins. This works, but it causes issues with the Git integration (see Section \ref{cloning} on how to fix this). One might think that we can add the module names to the build file, but in our case that did not help solve the issue.

If the Plugin folder is not present, make it yourself.

\subsection{Datatable bug}\label{datatableBug}
Previously, we stored ``blendshape source" data, ``blendshape target" data, ``weight" data, and ``multiple blendshapes to drive" data separately within maps and arrays in blueprints. This is very redundant so we decided to move these structures to be encapsulated in a single C++ Struct. After coding this in C++ and creating a datatable out of this in Unreal Engine, we populated it with data (an example of a single entry can be seen in Figure \ref{fig:blendshapeStructExample}).

\begin{figure}[hbt!]
    \centering
    \includegraphics[width=\textwidth]{imgs/singleBlendshapeData.png}
    \caption{An example source blendshape that drives a single target blendshape with a weight of 1.}
    \label{fig:blendshapeStructExample}
\end{figure}

This worked very well until we closed the editor, re-opened it and tried to view the datatable again. We get prompted by Unreal with an Error message. Pressing "Yes" at error message will open the asset without any of the struct information. Where did our data go? After a lot of fighting, we broke the entire project again and we had to rebuild through our IDE. Somehow that fixed the issue for the datatable as wel. Later into development the table was not useable in a blueprint, however. After further inspection we noticed that the ``RowStruct" of the datatable changed to something prepended with ``LIVECODE". Therefore we concluded that the entire problem was caused by the hotloading functionality of Unreal's compilation system. Disabling this functionality fixed the problem (more on best practices for compiling in Section \ref{CompilingAndBuilding}).

\subsection{Exporting/importing Datatables}
When exporting a datatable, you might be tempted to export it the same way you would export any other asset. That doesn't work however. Simply export it as csv or json in order to import it in a new project. In the new project, create a new datatable, open it, and use the reimport button to import the csv or json.

\subsection{Layering/blending animations with blendshapes}
What if we want to take a face animation and combine it with a body animation? If we simply put both animations over each other it in the level sequencer we get weird results. It's almost as if all animations are halved. This is due to the fact that we are blending a static a-pose (coming from the face animation) with the body animation.

\textit{Fix:} use the technique from section \ref{sec:bonegrouping}. Make an animation slot that only contains a leaf bone and bind the face animation to this slot.

\subsection{Python tick not saving file?}
See section \ref{PyUE} for more details.

\subsection{Curve node not animating morph targets}
Did you make sure that you enabled the materials to be used with morphtargets (\url{https://forums.unrealengine.com/t/material-needs-morph-support/12355})?

Lastly, it is important that the skeleton has the curves to control the morphtargets. If the curves are missing the animation will not be able to control the morphtargets. When the curves are missing we see figure \ref{fig:noCurves}. What we want to see is figure \ref{fig:curves}. The curve names should be the same as the blendshape names of the avatar. \textit{Tip:} instead of adding them by hand, copy paste them over from an avatar that already has them. If the names need to be changed, then alter them. I want to do this automatically, but I've yet to find a way (keep track of forum post for updates on this: \url{https://forums.unrealengine.com/t/how-to-add-curves-on-a-skeletal-mesh-programatically/2332340}). Lastly, I've asked why we even need curves for morphtargets in the following post: \url{https://forums.unrealengine.com/t/why-do-we-need-animation-curves-on-the-skeleton-for-controlling-morph-targets/2374534}.
\begin{figure}[hbt!]
    \centering
    \includegraphics[width=\textwidth]{imgs/nocurves.png}
    \caption{No curves on a skeleton.}
    \label{fig:noCurves}
\end{figure}
\begin{figure}[hbt!]
    \centering
    \includegraphics[width=\textwidth]{imgs/curves.png}
    \caption{What we expect, correct curves on the skeleton.}
    \label{fig:curves}
\end{figure}

%-------------------------------------------------------------------------------
%	REFERENTIES
%-------------------------------------------------------------------------------

\printbibliography

%-------------------------------------------------------------------------------
%	BIJLAGEN 
%-------------------------------------------------------------------------------

%TC:ignore
% \appendix 
% \section{Bijlage {\LaTeX} code}
% Bijgevoegd zijn de \textattachfile{main.tex}{code} en 
% \textattachfile{references.bib}{bibliografie}.
%TC:endignore

%-------------------------------------------------------------------------------
\end{document}